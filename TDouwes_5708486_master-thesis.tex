\documentclass[11pt,a4paper]{report}
\usepackage[latin2]{inputenc}
\usepackage[english]{babel}
\usepackage{amsmath}
\usepackage{amsfonts}
\usepackage{amssymb}
\usepackage{amsthm}
\usepackage{mathrsfs}
\usepackage{mathtools}
\usepackage[shortlabels]{enumitem}
\usepackage{physics}
\usepackage{siunitx}
\usepackage{witharrows}
\usepackage{xifthen}
\usepackage[margin=1cm]{caption}
\usepackage{braket}
\usepackage{bm}
\usepackage{geometry}
\usepackage{xcolor}
\usepackage{tikz-cd}
\usepackage{fancyhdr}
\pagestyle{fancy}
\usepackage[style=alphabetic]{biblatex}
\addbibresource{references.bib}
\usepackage[hidelinks]{hyperref}
\usepackage{tocloft}

\newtheorem{theorem}{Theorem}[section]
\newtheorem{proposition}[theorem]{Proposition}
\newtheorem{lemma}[theorem]{Lemma}
\newtheorem{corollary}[theorem]{Corollary}

\theoremstyle{definition}
\newtheorem{example}[theorem]{Example}
\newtheorem{definition}[theorem]{Definition}

\theoremstyle{remark}
\newtheorem*{remark}{Remark}


\DeclareMathOperator{\Vol}{Vol}
\DeclareMathOperator{\pr}{pr}
\DeclareMathOperator{\On}{O}
\DeclareMathOperator{\SO}{SO}
\DeclareMathOperator{\U}{U}
\DeclareMathOperator{\SU}{SU}
\DeclareMathOperator{\id}{id}
\DeclareMathOperator{\Curves}{Curves}
\DeclareMathOperator{\standard}{standard}
\DeclareMathOperator{\gl}{GL}
\DeclareMathOperator{\Ad}{Ad}
\DeclareMathOperator{\ad}{ad}
\DeclareMathOperator{\dom}{dom}
\DeclareMathOperator{\Alt}{Alt}
\DeclareMathOperator{\diag}{diag}

\renewcommand{\thefootnote}{\fnsymbol{footnote}}

\newcommand{\DD}{{\rm D}}
\renewcommand{\qed}{%
	\ifmmode\tag*{$\square$}
	\else\hfill$\square$
	\fi}
\newcommand{\?}{\stackrel{?}{=}}
\newcommand{\e}[1][]{{\rm e}^{#1}}
\newcommand{\N}{\mathbb{N}}
\newcommand{\Z}{\mathbb{Z}}
\newcommand{\R}[1][]{\mathbb{R}^{#1}}
\newcommand{\C}[1][]{\mathbb{C}^{#1}}
\newcommand{\F}{\mathbb{F}}
\newcommand{\Mn}[1][\C]{\mathcal{M}_n(#1)}
\newcommand{\GLn}[1][\C]{\gl_n(#1)}
\newcommand{\GL}[1][n]{\gl_{#1}(\C)}
\newcommand{\Pn}[1][n]{\mathbb{P}^{#1}}
\newcommand{\Lie}[1]{\mathcal{L}_{#1}}
\newcommand{\la}[1][g]{\mathfrak{#1}}
\newcommand{\vct}{\mathfrak{X}}
\newcommand{\fund}[1]{#1^\#}
\newcommand{\Lg}{\mathscr{L}}
\newcommand{\Ac}{\mathcal{A}}
\newcommand{\Fc}{\mathcal{F}}
\newcommand{\Gc}{\mathcal{G}}
\newcommand{\Rc}{\mathcal{R}}
\newcommand{\Oc}{\mathcal{O}}
\newcommand{\Cinf}{C^\infty}
\newcommand{\cov}[1]{\nabla_{\!#1}}
\newcommand{\inner}[2]{\langle #1, #2 \rangle}
\newcommand{\onto}{\hookrightarrow}
\newcommand{\supto}{\supset\kern-1.7pt\to}
\newcommand{\lowint}{\mkern3mu\underline{\vphantom{\intop}\mkern7mu}\mkern-10mu\int}
\newcommand{\upint}{\mathchoice%
	{\mkern13mu\overline{\vphantom{\intop}\mkern7mu}\mkern-20mu}%
	{\mkern7mu\overline{\vphantom{\intop}\mkern7mu}\mkern-14mu}%
	{\mkern7mu\overline{\vphantom{\intop}\mkern7mu}\mkern-14mu}%
	{\mkern7mu\overline{\vphantom{\intop}\mkern7mu}\mkern-14mu}%
	\int}
\newcommand{\tran}{^{\mkern-1.5mu\mathsf{T}}}
\newcommand{\herm}{^\dag}
\newcommand{\ext}[1]{\widetilde{#1}}
\newcommand{\argdot}{\makebox[1ex]{\textbf{$\cdot$}}}

\renewcommand{\phi}{\varphi}
\renewcommand{\O}{\On}

\newcommand{\red}[1]{\textcolor{red}{#1}}
\newcommand{\blue}[1]{\textcolor{blue}{#1}}

%\title{Formalising local symmetries: An introduction to mathematical gauge theory}
%\author{Thijs Douwes}

\begin{document}
	
\begin{titlepage} % Suppresses displaying the page number on the title page and the subsequent page counts as page 1
		\newcommand{\HRule}{\rule{\linewidth}{0.5mm}} % Defines a new command for horizontal lines, change thickness here
		
		\center % Centre everything on the page
		
		%------------------------------------------------
		%	Headings
		%------------------------------------------------
		
		\textsc{\LARGE Utrecht University}\\[1.5cm] % Main heading such as the name of your university/college
		
		%------------------------------------------------
		%	Title
		%------------------------------------------------
		
		\HRule\\[0.4cm]
		
		{\Huge\textsc{Topology of Weyl-semimetals}\\[.2cm] \huge  with non-orientable Brillouin zones}\\[0.4cm] % Title of your document
		
		\HRule\\[.8cm]
		
		
		{\large by}\\[.8cm]
		
		
		{\LARGE Thijs \textsc{Douwes}}\\[1.1cm]
		
		
		
		{\Large \textsc{A thesis}}\\[.8cm]
		
		
		{\large Submitted to the Department of Physics}\\[.1cm]
		{\large in partial fulfilment of the requirements}\\[.1cm]
		{\large for the degree of}\\[.5cm]
		
		{\Large Master of Science}\\[.9cm]
		
		
		{\large under the joint supervision of}\\[1cm]
		
		
		\begin{minipage}{0.45\textwidth}
			\begin{flushleft}
				\large
				%			\textit{Supervisor}\\
				Prof. Cristiane \textsc{de Morais Smith}\\
				\normalsize Department of Physics
			\end{flushleft}
		\end{minipage}
		~
		\begin{minipage}{0.45\textwidth}
			\begin{flushright}
				\large
				%			\textit{Supervisor}\\
				Dr. Marcus \textsc{St{\aa}lhammar}\\
				\normalsize Department of Physics
			\end{flushright}
		\end{minipage}
		
		\vfill\vfill\vfill % Position the date 3/4 down the remaining page
		
		{\large October 2024} % Date, change the \today to a set date if you want to be precise
		
		
		\vfill % Push the date up 1/4 of the remaining page
		
	\end{titlepage}
	%\maketitle %TODO title page
\pagenumbering{roman}
\setcounter{page}{2}
	
	
\chapter*{Abstract}
	
\tableofcontents
	
\newpage
\pagenumbering{arabic}
	
\chapter{Introduction}

Example citation.\cite{einstein_rel} Example expanded citation.\parencite[Theorem 5.6]{lee_manifolds}

\section{Main results}

\section{Overview}

\section{Prerequisites}

\section{Notational conventions}
	
	
\chapter[Topological states of matter]{Topological states of matter \& symmetries}	

	
\chapter{Weyl semimetals}

\section{Physics perspective}

\section{Mathematics perspective}


\chapter{Non-orientable manifolds}

\section{Mathematical exploration}

\section{Physical implications}

\appendix 
\cftaddtitleline{toc}{chapter}{APPENDICES}{}

\chapter{Homology and cohomology}

\printbibliography
	
	
\end{document}