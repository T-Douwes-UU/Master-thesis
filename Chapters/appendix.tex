\cftaddtitleline{toc}{chapter}{$\qquad$APPENDICES}{}
\chapter{Homology and cohomology}

The concepts of homology and its counterpart cohomology are indispensable in algebraic topology.  %TODO

Here we offer a brief introduction to these concepts, aimed at the uninitiated physicist. The goal here is not to be completely rigorous, but to give a sufficiently complete understanding that the applications discussed in the main text may be understood in their proper context. For a more complete picture, the interested reader is referred to standard texts in algebraic topology such as \cite{Hatcher_algebraic-topology} and \cite{Bredon_topo-geometry}. A more geometric treatment is also found in \cite{Bott-Tu_differential-forms}.


\section{Homotopy}

%TODO


\section{Homology}

Suppose we have some topological space---for example, the torus $\T^2$---and we want to study %TODO

The basic idea underlying homology is that information about the topology of a space can be gained from studying non-trivial subspaces. In the related \emph{homotopy} theory, this is achieved by mapping $n$-dimensional spheres into the space, and seeing whether or not they can be contracted to a point.


\section{Cohomology}