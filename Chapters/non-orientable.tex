\chapter{Non-orientable manifolds}

\section{Mathematical exploration}

{\color{red}
Concepts explored in personal notes so far:
\begin{itemize}
	\item Calculations of (co)homology and semimetal MV sequence for manifolds in $\geq2$ dimensions:
	
		\subitem All compact surfaces without boundary, i.e.\ the surfaces $M_g$ and $N_g$
		
		\subitem All spaces of the form $M = K^2 \times \T^{d-2}$
	
	\item The map $\Sigma:H^{d-1}(\bigsqcup_{k}S^{d-1})\to H^d(M)$ in the semimetal MV has a clear interpretation in terms of total charge in the (orientable) $d=3$ case. This would provide a clear picture of the total charge cancellation in the orientable case ($H^d(M) = \Z$ in general) vs. the mod 2 charge cancellation in the non-orientable case ($H^d(M) = \Z_2$ in general).
	
	\item However, $\Sigma$ and the other maps in the MV sequence are difficult to interpret in the $\chi\neq 0$ case (maybe even generally for odd dimensions). Taking the oriented case as an example, the MV sequence ends as
	\begin{align*}
		H^{d-1}(M\setminus\Delta)\ \rightarrow\ H^{d-1}\left(\bigsqcup_{k}S^{d-1}\right) \cong \Z^k\ \overset{\Sigma}{\rightarrow}\ H^d(M) \cong \Z
	\end{align*}
	so that the ``charge configuration'' in $\Z^k$ must map to 0 by $\Sigma$ in order to descend from the semimetal, regardless of whether $\chi=0$.
	
	\item This may imply that the Bloch vector field carries more topological information about the total charge than the MV sequence (which makes sense since it generates \emph{all} homology groups of the valence bundle, and all Betti numbers factor into $\chi$). As a concrete example, consider $M=S^2$ with a single puncture of charge $+2$. The punctured sphere is topologically a disc, so that the valence bundle must be trivial, while the Bloch vector field is topologically non-trivial in the sense that it has an index $+2$ singularity. In addition, all relevant $H_n(A)\oplus H_n(B)$ are zero, so that the semimetal MV reduces to the statement that $H_2(S^2)\cong H_1(S^1)$.
	
	\item It may even be the case that the valence bundle cannot be generated from the Bloch vector field in the $d=2$ case; it's probably worth studying the $d\in\set{3,4,5}$ cases (pullback of some universal bundle) to learn more about this. The $d=3$ case should be especially helpful in understanding how the valence bundle arises from the vector field.
	
	\item A complicating factor in the non-orientable case is that the homology groups are different from the cohomology groups, since the torsion moves up one dimension. This makes the homological semimetal MV different from the cohomological one (it's a short exact sequence in $d\geq3$!), and this leads to additional challenges in interpretation.
	
	\item The map $H: \R^3\to\la[su](2),\ \vec{h}\mapsto \vec{h}\cdot\vec{\sigma}$ is an isomorphism of Lie algebras, with the cross product as a Lie bracket on $\R^3$. Still the vector field is discontinuous on a non-orientable manifold, while $H$ is not. This suggests an alternative approach for constructing the valence bundle: consider $h$ as a map $M\to\R^d$ instead of an element of $\vct(M)$, and then pull back the universal bundle along the unit map $\hat{h}:M\setminus\Delta\to S^{d-1}$. That is, we detach $\vec{h}$ from the tangent bundle and consider it a more abstract map. An added ``benefit'' of this is that we lose all coordinate dependence. However, this is also a downside in the sense that the map will not be subject to the same constraints (Poincaré--Hopf etc.) that the vector field is; for example, $S^2\to\R^2,\ x\mapsto(1,0)$ is a perfectly valid map that would violate the hairy ball theorem as a vector field (and this is a result of being unable to cover $S^2$ by a single chart). At this point the question may become more about which description is more physical in nature, and the non-orientable Weyl point paper\cite{Foncesca-Vaidya_nonorientable} seems to imply there may be more to the $h:M\to\R^3$ story.
\end{itemize}
}  % end red colour

\section{Physical implications}