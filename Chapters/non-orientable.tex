\markedchapter{Non-orientable mfds}{Non-orientable manifolds}

Symmetries play a crucial role in establishing and differentiating topological properties of physical systems; this much is already evident from the tenfold way classification discussed in Section \ref{sec:symm-classes}. However, in recent years researchers have begun to recognise that symmetries beyond the standard time reversal, charge conjugation and chiral symmetries can give rise to distinct topological invariants that are not fully captured by this classification.

An important class of these extended symmetries is comprised by the space groups acting on periodic lattices. These impose additional structure on the unit cell of the lattice, such as rotational or reflection symmetry. Of special interest here are the so-called non-symmorphic space group symmetries, which combine basic rotation, reflection or inversion with a lattice translation. These symmetries have no fixed points and may induce novel topological states when applied to a material lattice.

Very recently, the feasibility of applying these non-symmorphic symmetries in momentum space has been demonstrated both theoretically and experimentally. This opens up new and interesting avenues of research relating to Brillouin zone topology. In particular, the Brillouin zone may become effectively non-orientable, challenging the notion of chirality for Weyl points.

We begin this chapter with a short review of existing literature surrounding these concepts, culminating in a treatment of a recent paper which applies them to Weyl semimetals. We then present a novel topological analysis of the system described in this paper, in terms of the cohomology and homology tools presented in the last chapter.

\markedsection{Review}{Review of recent literature}

{\color{blue}
\begin{itemize}
	\item Chen, Yang and Zhao describe a 2D system in \cite{CYZ_Klein-gauge} where the Brillouin zone has the topology of a Klein bottle. This system features a $\Z_2$ invariant, as opposed to a $\Z$ Chern number, relating to the fact that $H^2(K^2) = \Z_2$.
	
	\item Other works have verified this experimentally and generalised the theory to higher dimensions and other non-symmorphic symmetries.
\end{itemize}
}

\subsection{Non-orientable Weyl semimetals}

{\color{blue}
\begin{itemize}
	\item Fonseca, Vaidya et al.\ report on the properties of Weyl points in a $K^2\times S^1$ Brillouin zone \cite{Fonseca-Vaidya_nonorientable}. This is achieved using a momentum space glide symmetry:
	\begin{equation}\label{eq:3D_glide}
		H(k_x, k_y, k_z) = H(-k_x, k_y + \pi, k_z).
	\end{equation}
	An $\R P^2\times S^1$ topology from a double glide symmetry is also discussed in the supplemental material.
	
	\item The fundamental Brillouin zone is parametrised with $k_x,k_z\in[-\pi,\pi]$ and $k_y\in[-\pi,0]$, with the necessary boundary identifications.
	
	\item Emphasis is placed on the "orientation-reversing planes" at $k_y = -\pi$ and $k_y = 0$: moving a Weyl point across one of these planes makes it return on the other side with opposite charge. As a result, there is no absolute notion of chirality on the Klein bottle.
	
	\item The authors still apply a notion of relative chirality, tied to the choice of fundamental domain. With respect to this notion, it is shown that Nielsen--Ninomiya is circumvented: The total chirality of the Weyl points in the fundamental domain may be non-zero.
	
	\item An argument is made relating the breaking of Nielsen--Ninomiya to a claimed discontinuity of the Bloch vector field defining the Hamiltonian.
	
	\item Weyl points with unit charge must still be connected via Fermi arcs; a pair of points with the same charge may be connected by a Fermi arc which crosses the orientation-reversing boundary an odd number of times.
	
	\item A $\Z_2$ invariant is shown to exist on $K^2$-like slices of the EBZ, and it is claimed that this invariant is sourced by the Weyl points.
	
	\item An experimental realisation of the system is demonstrated using 1D photonic crystals with two synthetic momenta.
\end{itemize}
}


\markedsection{Topology}{Topological exploration}

The purpose of this section is to reframe and analyse the non-orientable Weyl semimetals described in \cite{Fonseca-Vaidya_nonorientable} in terms of the algebraic topology language from the previous chapter. This approach has the advantage of being coordinate free, and as such it provides a more fundamental understanding of the system's topological properties. We obtain a direct description of how the Nielsen--Ninomiya theorem is modified into a $\Z_2$ charge cancellation condition. We also obtain a more complete picture of the different invariants associated with such a system, and are able to distinguish which are related to the topological insulator phase, and which relate to the introduction of Weyl nodes. To the knowledge of the author, the insights contained in this section are novel.

\subsection{Motivation}

In order to motivate the proposed coordinate-free description, we begin by clearing up some minor points of confusion present in \cite{Fonseca-Vaidya_nonorientable}.

To begin with, there is a relatively strong emphasis on the two ``orientation-reversing planes'' at $k_y=\pm\pi$ and $k_y=0$. For example, it is stated that relative chirality can be defined unambiguously on fundamental domains that avoid these planes. This is true on a technical level, but it creates the impression that the orientation reversal occurs locally at the boundary of the fundamental domain, raising questions about the nature of Weyl points existing on these planes.

In reality, orientation reversal is a global feature. We are free to reparametrise the fundamental domain in a way that includes the planes $k_y=\pm\pi$ and $k_y=0$, and the notion of relative chirality may change as a result; this is illustrated in Figure \ref{fig:BZ_param}.
\begin{figure}[htb!]
	\centering
	\subcaptionbox{$-\pi \leq k_y \leq 0$\label{subfig:BZ_basic}} {\includegraphics[width=.3\textwidth]{Images/BZ_basic}}
	\hfil
	\subcaptionbox{$-\pi/2 \leq k_y \leq \pi/2$\label{subfig:BZ_mid}} {\includegraphics[width=.3\textwidth]{Images/BZ_mid}}
	\hfil
	\subcaptionbox{\label{subfig:BZ_left}}{\includegraphics[width=.3\textwidth]{Images/BZ_left}}
	\caption{Top view of the 3D Brillouin torus (or 2D surface torus) for a given Weyl semimetal state obeying the glide symmetry in Equation \eqref{eq:3D_glide}, with Weyl points and oriented Dirac loops (Fermi arcs) drawn in. Different parametrisations of the fundamental domain are shaded in teal: (a) the domain outlined in \cite{Fonseca-Vaidya_nonorientable}; (b) the same domain shifted in the $k_y$ direction; (c) a domain spanning the $k_y$ direction. Each of these domains is homeomorphic to $K^2\times S^1$ ($K^2$) under the boundary identifications shown. Note that the relative chirality of the two Weyl nodes in the fundamental domain changes under both alternative parametrisations.}
	\label{fig:BZ_param}
\end{figure}
For any given set of distinct Weyl points obeying the symmetry, it is possible in principle to achieve any relative chirality by reparametrising the fundamental domain. \red{This may mean that the charge of the Weyl points is fundamentally an $\N$ invariant on $K^2\times S^1$, but this is not clearly reflected in the cohomology description.}

%TODO

{\color{blue}
\begin{itemize}		
	\item The description of breaking Nielsen--Ninomiya in terms of a discontinuous vector field is somewhat suspect from a mathematical viewpoint; the Bloch vector field should really be taken as a map $K^2\times S^1\to\R^3$ in this case, not as a section of the tangent bundle. A proper description should come from the underlying cohomology.
	
	\item The $\R P^2$ case is more complicated mathematically than is acknowledged in \cite{Fonseca-Vaidya_nonorientable}, as the total group action is not free in this case. 
	
	\item Non-orientability of the Brillouin zone forces us to decide between twisted homology and twisted cohomology as a proper description (cf. TRS-invariant Weyl semimetals to be described in Section \ref{sec:T-WSMs}; in that case the cohomology is twisted.)
	
	\item The sign of the Chern number respects the symmetry, but the orientation of Dirac strings does not, making twisted homology the natural choice in this instance. [Make a nice figure demonstrating this clearly]. This makes the description fundamentally different from that of T-WSMs.
	
	\item On the level of vector bundle classifications, we are classifying complex line bundles over the EBZ, since normal (untwisted) cohomology is used. This agrees well with the K-theory insight that for a free group action, equivariant bundles are equivalent to normal bundles on the quotient space \parencite[Prop. 2.1]{Segal_K-theory}.
	
	\item Cohomology is easy to compute, but its dual twisted homology offers a very intuitive perspective of the invariants, demonstrating how the $\Z^3$ invariant on a 3D Chern insulator gets reduced to a $\Z\oplus\Z_2$ under this symmetry. The $\Z_2$ factor is precisely the $\Z_2$ invariant discussed by Fonseca, Vaidya et al. This intuition is especially clear when viewed as twisted equivariant homology on $\T^3$. [Include pictures of the action of the symmetry on the three basic homology invariants on $\T^3$.]
	
	\item Under this description, the semimetal Mayer--Vietoris sequence becomes
	\begin{equation*}
		0\to H^2(K^2\times S^1) \to H^2(K^2\times S^1\setminus W) \to H^2(S_W) \to H^3(M) \to 0,
	\end{equation*}
	which can be calculated explicitly (using e.g. cellular (co)homology) as
	\begin{equation*}
		0\to \mathbb{Z}\oplus\mathbb{Z}_2 \to \mathbb{Z}^{1+k}\oplus\mathbb{Z}_2 \to \mathbb{Z}^k \to \mathbb{Z}_2 \to 0.
	\end{equation*}
	From the $\Z_2$ in the last term, we immediately recover the $\Z_2$ charge cancellation no-go theorem. It also becomes clear that a single Weyl point may carry non-trivial topology (i.e. already adds a factor to the group of invariants).
\end{itemize}

Additionally:

\begin{itemize}
	\item More elementary systems such as 3D type A with inversion symmetry also exhibit non-orientable EBZs, but in this case the description is complicated by the existence of fixed points (the TRIM).
	
	\item Applicability of this description is probably limited under addition of non-trivial additional bands, e.g. 4-band models incorporating spin and orbital degrees of freedom. The full scope of applicability is somewhat of an open question at this point. (E.g., how well are 2D chiral WSMs described by this homology picture?)
\end{itemize}
}



{\color{red}
\section*{Notes}
Concepts explored in early personal notes:
\begin{itemize}
	\item Calculations of (co)homology and semimetal MV sequence for manifolds in $\geq2$ dimensions:
	\begin{itemize}
		\item All compact surfaces without boundary, i.e.\ the surfaces $M_g$ and $N_g$
		
		\item All spaces of the form $M = K^2 \times \T^{d-2}$
	\end{itemize}
	
	\item The map $\Sigma:H^{d-1}(\bigsqcup_{k}S^{d-1})\to H^d(M)$ in the semimetal MV has a clear interpretation in terms of total charge in the (orientable) $d=3$ case. This would provide a clear picture of the total charge cancellation in the orientable case ($H^d(M) = \Z$ in general) vs. the mod 2 charge cancellation in the non-orientable case ($H^d(M) = \Z_2$ in general).
	
	\item However, $\Sigma$ and the other maps in the MV sequence are difficult to interpret in the $\chi\neq 0$ case (maybe even generally for odd dimensions). Taking the oriented case as an example, the MV sequence ends as
	\begin{align*}
		H^{d-1}(M\setminus\Delta)\ \rightarrow\ H^{d-1}\left(\bigsqcup_{k}S^{d-1}\right) \cong \Z^k\ \overset{\Sigma}{\rightarrow}\ H^d(M) \cong \Z
	\end{align*}
	so that the ``charge configuration'' in $\Z^k$ must map to 0 by $\Sigma$ in order to descend from the semimetal, regardless of whether $\chi=0$.
	
	\item This may imply that the Bloch vector field carries more topological information about the total charge than the MV sequence (which makes sense since it generates \emph{all} homology groups of the valence bundle, and all Betti numbers factor into $\chi$). As a concrete example, consider $M=S^2$ with a single puncture of charge $+2$. The punctured sphere is topologically a disc, so that the valence bundle must be trivial, while the Bloch vector field is topologically non-trivial in the sense that it has an index $+2$ singularity. In addition, all relevant $H_n(A)\oplus H_n(B)$ are zero, so that the semimetal MV reduces to the statement that $H_2(S^2)\cong H_1(S^1)$.
	
	\item It may even be the case that the valence bundle cannot be generated from the Bloch vector field in the $d=2$ case; it's probably worth studying the $d\in\set{3,4,5}$ cases (pullback of some universal bundle) to learn more about this. The $d=3$ case should be especially helpful in understanding how the valence bundle arises from the vector field.
	
	\item A complicating factor in the non-orientable case is that the homology groups are different from the cohomology groups, since the torsion moves up one dimension. This makes the homological semimetal MV different from the cohomological one (it's a short exact sequence in $d\geq3$!), and this leads to additional challenges in interpretation.
	
	\item The map $H: \R^3\to\la[su](2),\ \vec{h}\mapsto \vec{h}\cdot\vec{\sigma}$ is an isomorphism of Lie algebras, with the cross product as a Lie bracket on $\R^3$. Still the vector field is discontinuous on a non-orientable manifold, while $H$ is not. This suggests an alternative approach for constructing the valence bundle: consider $h$ as a map $M\to\R^d$ instead of an element of $\vct(M)$, and then pull back the universal bundle along the unit map $\hat{h}:M\setminus\Delta\to S^{d-1}$. That is, we detach $\vec{h}$ from the tangent bundle and consider it a more abstract map. An added ``benefit'' of this is that we lose all coordinate dependence. However, this may also be a downside in the sense that the map will not be subject to the same constraints (Poincaré--Hopf etc.) that the vector field is; for example, $S^2\to\R^2,\ x\mapsto(1,0)$ is a perfectly valid map that would violate the hairy ball theorem as a vector field (and this is a result of being unable to cover $S^2$ by a single chart). At this point the question may become more about which description is more physical in nature, and the non-orientable Weyl point paper\cite{Fonseca-Vaidya_nonorientable} seems to imply there may be more to the $h:M\to\R^3$ story. It also seems to agree better with the intuition of an applied external potential removing all Weyl nodes -- something that's impossible for $\chi\neq0$ if charge corresponds to vector field index. It also explains how the valence bundle can be trivial on the once punctured $S^2$.
	
	\item In light of the previous point, this may be an important observation: every $d$-manifold $M$ with $\chi(M)=0$ admits a nowhere-vanishing vector field (\href{https://math.stackexchange.com/questions/47370/if-a-manifold-m-has-zero-euler-characteristic-there-is-a-non-vanishing-vector-f}{link}). \st{This may imply that the vector field description is equivalent to the map to $\R^d$ in these cases, though one needs to be careful about charts. It would be good to find or write a (dis)proof for something like $\vct(M)\cong\Cinf(M,\R^d)$ (or similar for non-vanishing maps) in this case. Or more specifically:}
	\[
		\st{\big[M\setminus\Delta,S^{d-1}\big] \stackrel{?}{\cong} \Set{\vec{h}\in\vct(M\setminus\Delta) | \text{$\vec{h}$ is non-vanishing}}}
	\]
	Update: I think the real requirement for equivalence is that the base manifold $M$ is parallelisable (i.e.\ has a trivial tangent bundle), since we're essentially using a trivial $\R^d$-bundle in this construction.
	
	\item Any smooth $d$-manifold can be given a CW complex structure with one $d$-cell (\href{https://mathoverflow.net/questions/120799/manifolds-admitting-cw-structure-with-single-n-cell}{link}). On this $d$-cell there is an exact correspondence between vector fields and maps to $\R^d$, since it can be embedded in $\R^d$. What distinguishes the two is how points on the boundary of the $d$-cell are identified with each other; this determines whether the ``vectors'' need to change orientation. To illustrate:
	\includegraphics[width=.9\textwidth]{Images/vectorfield-vs-map}
	
	\item On any orientable manifold, the Stokes' theorem argument shows that the total charge must be zero regardless of Euler characteristic:
	\[
		\sum_{\alpha}w(S_\alpha) = \sum_{\alpha}\int_{S_\alpha} c_1(E) = \sum_{\alpha}\int_{S_\alpha}\frac{\Tr\Fc}{2\pi} = \int_{B'} \dd{\frac{\Tr\Fc}{2\pi}} = 0
	\]
	where the last equality holds by the Bianchi identity for the trace. This means the valence bundle cannot be a pullback along a tangent vector field for $\chi\neq0$.
	
	On a non-orientable manifold, this argument doesn't hold since the integral over $B'$ isn't well defined.
	
	\item Total chirality isn't well defined on a non-orientable manifold (at least in odd dimensions, not sure how to interpret even dimensions). Still there is charge cancellation in the form of Fermi arcs etc.; it may take moving to a different homology system to get the full picture, such as homology with local coefficients or equivariant homology. (See e.g. \cite{Thiang_equivariant})
	
	\item It may be worth classifying which manifolds are candidates for physical material Brillouin zones; I have a feeling that this might be restricted to those manifolds for which the $n$-torus is a covering space. In this case a full classification of symmetries on the torus (and e.g. their related equivariant homologies) would be sufficient to classify all material topologies. This classification is related to space group symmetries.
\end{itemize}
}  % end red colour