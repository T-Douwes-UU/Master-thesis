\chapter*{Abstract}

Weyl semimetals (WSMs) are a class of 3D materials that feature point-like band crossings in momentum space. These crossings are called Weyl points, since they represent Weyl fermion-like chiral modes in the material. In a generic WSM, the so-called Nielsen–Ninomiya theorem restricts the chiralities of these Weyl points to add to zero on the momentum space unit cell—commonly referred to as the Brillouin zone. This cancellation prevents global chiral anomalies from appearing in a material.

However, recent work indicates that the Nielsen–Ninomiya theorem is circumvented when the Brillouin zone is non-orientable—a condition which can be physically realised under certain symmetries. For example, two isolated Weyl points with the same charge may appear in this scenario.

The aim of this thesis is to shed light on this and other features of non-orientable WSMs. We do this by extending an existing algebraic topology framework, which studies WSMs in terms of cohomology and homology, to the non-orientable case. This allows the physics of these systems to be studied in a more rigorous, coordinate-free setting. In particular, we are able to pinpoint the mechanism behind the circumvention of the Nielsen–Ninomiya theorem and assess its physical consequences. In addition to this, we generalise the setting to previously unstudied non-orientable Brillouin zones and other forms of orientation-reversing symmetry.