\chapter{Conclusions and outlook}\label{chap:conclusions}

In this chapter, we take stock of the current state of the research performed in this work. We begin in Section~\ref{sec:summary} by briefly summarising the key results presented in Chapter~\ref{chap:non-orientable}. We then expand on some avenues of possible further research in Section~\ref{sec:outlook}, speculating on specific lines of inquiry where appropriate.

\markedsection{Summary}{Summary of main results}\label{sec:summary}

The central results of this thesis are focused around gaining a deeper understanding of how non-orientability in momentum space alters the topological description of Weyl semimetals---a state of matter that is fundamentally chiral in nature. This was achieved in large part by applying the appropriate tools from algebraic topology to a glide symmetric system featuring a Brillouin zone with $\KS$ topology, of which some important properties had already been established in Ref.~\cite{Fonseca-Vaidya_nonorientable}.

\subsubsection{Contextualisation of the Poincaré--Hopf theorem}

An important aspect of obtaining a cogent topological description lies in deciding which tools from topology are suitable to use in the first place. To this end, we have placed the role that the Poincaré--Hopf theorem plays in deriving the Nielsen--Ninomiya charge cancellation on firmer theoretical footing. In three dimensions, Poincaré--Hopf states that the singularities of a tangent vector field on a compact manifold should have topological indices which sum to zero. Nielsen--Ninomiya is then usually derived by noting that the $\h(\k)$ appearing in the model Hamiltonian can be considered such a tangent vector field. We note that this is categorically not the case on non-orientable manifolds; in this context, $\h$ is a section of the trivial $\R^3$-bundle over the manifold, rather than the tangent bundle on which tangent vector fields are defined. The mismatch in orientability between the base manifold and the $\R^3$-bundle is precisely what causes the Poincaré--Hopf/Nielsen--Ninomiya theorems to become modified to a mod 2 charge cancellation.

\subsubsection{Formalism for obtaining twisted coefficients}

In order to obtain a full classification in terms of homology and cohomology on a non-orientable system, one needs to decide how to restore Poincaré duality. This is done by twisting the coefficients of either the homology or the cohomology; only one of these choices yields accurate topological invariants. The correct twisting is usually motivated using arguments relating to vector bundle classification. These arguments are relatively tractable for a free unitary action such as the one featured in the $\KS$ system, but they quickly become conceptually involved when studying more complex symmetries (i.e. those that have fixed points, are anti-unitary, etc.), requiring much mathematical legwork to be done in order to obtain the final topological invariants.

To remedy this situation, we have demonstrated that a highly straightforward heuristic arises in the specific context of Weyl semimetals: one can study how a given orientation-reversing symmetry acts on Weyl points. If the symmetric partner of a Weyl point has the same chirality, this indicates a twist in the cohomology, and if it has the opposite chirality, the twist is in the homology. This twist then carries over to the underlying insulating topology, so that this heuristic is also useful for classifying those insulating phases which are mediated by transitional Weyl semimetals. In particular, we demonstrate that orientation-reversing unitary symmetries must always be described using twisted homology.

This formalism does not always uniquely identify the correct classifying groups; if the symmetry is not free, one needs to study how its fixed points are treated in the equivariant homology and cohomology. Nevertheless, these features may also be assessable using heuristic approaches, depending on the system. In any case, the formalism is useful for developing physical intuition, and it serves as a good sanity check in classification schemes making use of these twisted groups---especially those involving anti-unitary symmetries.

\subsubsection{Classification of $\KS$ bulk and $K^2$ surface invariants}

We have provided a full classification and interpretation of the topological invariants on the $\KS$ system arising from a single momentum space glide symmetry, based on the explicitly calculated Mayer--Vietoris sequence in Equation~\eqref{eq:explicit-sequence-nonorientable}. This includes both the insulating invariants, which are classified by $\Z\oplus\Z_2$, and the semimetal invariants, classified by $\Z\oplus\Z_2\oplus\Z^k$ with $k$ the number of Weyl points on $\KS$. Aside from the $\Z_2$ invariant $\nu_z$ which has been well studied previously, we have also described a seemingly novel $\Z$ invariant $\nu_x$. We have demonstrated how $\nu_x$ and $\nu_z$ can be obtained from the action of the glide symmetry on two of the basic homology invariants in $H_1(\T^3)\cong\Z^3$, and how the third invariant is trivialised by the same action. Based on these observations, we have provided two integral equations for explicit calculation of $\nu_x$: Equation~\eqref{eq:z-invariant1} in terms of a simple Chern number in the insulating context, and Equation~\eqref{eq:z-invariant2} in terms of curved planes in the semimetallic context. This illustrates the usefulness of the twisted homology point of view.

We have also provided a classification of the surface states arising from truncation in the periodic $S^1$ direction, in the form of the Mayer--Vietoris sequence in Equation~\eqref{eq:K2-MV-explicit}. This sequence has very similar features to the bulk $\KS$ sequence, save for the fact that the $\Z_2$ invariant $\nu_z$ is projected out.

\subsubsection{Clarification of mod 2 charge cancellation}

The Mayer--Vietoris sequence in Equation~\eqref{eq:explicit-sequence-nonorientable} also provides a more solid conceptual basis for the mod 2 charge cancellation on $\KS$ described in Ref.~\cite{Fonseca-Vaidya_nonorientable}. It relates to the fact that the charges of all Weyl points must sum to zero in the group $H^2(\KS)\cong\Z_2$, which appears where a $\Z$ group would usually be in the orientable case. This is closely related to the lack of a canonical orientation at each Weyl point. In particular, it follows that the total chirality is ill defined as an integer, being sensitive to arbitrary local choices of orientation (or equivalently, choice of a fundamental domain). We assert that this amounts to an unphysical degree of freedom, so that any apparent net non-zero chirality should have no phenomenological consequences.

\subsubsection{Classification of non-orientable Brillouin zones}

Through use of the theory of space groups, we have ascertained that there are exactly three other symmetry groups in three dimensions besides the single glide symmetry which have a free action giving rise to a non-orientable fundamental Brillouin zone: $Cc$, $Pca2_1$ and $Pna2_1$. Just as for $\KS$, we have provided explicitly calculated Mayer--Vietoris sequences for each of these three systems. Each of them features the same $\Z_2$ charge cancellation as $\KS$; the differences lie in the insulating invariants. Explicitly, the insulating phases under $Cc$ symmetry are classified by $\Z\oplus\Z_2^2$, those under $Pca2_1$ are given by $\Z_2^2$ and those under $Pna2_1$ are given by $\Z_4$. This latter group is especially interesting, featuring a topological invariant of order four.

\subsubsection{Classification of inversion symmetric invariants}

Finally, we have used a heuristic ansatz to derive a Mayer--Vietoris sequence for Weyl semimetals with inversion symmetry. These materials comprise an important class of experimentally verified magnetic Weyl semimetals. The Mayer--Vietoris sequence given in Equation~\eqref{eq:inversion-MV-explicit} correctly predicts that no form of charge cancellation occurs within the effective Brillouin zone, and it classifies a set of insulating invariants that agrees precisely with existing works. Consequently, our ansatz leads to the correct classification of semimetallic invariants.

Importantly, this demonstrates that there are cases in which a system with high-symmetry points can nonetheless be correctly classified using relatively elementary tools such as ordinary cellular homology. This alleviates the need to resort to more conceptually involved arguments relying on equivariant cohomology and the related vector bundle classification, which often lead to challenging calculations.


\markedsection{Recommendations}{Research recommendations}\label{sec:outlook}

We close this thesis by making an inventory of the open questions that remain, along with some possible applications and experimental realisations.

The biggest caveat that comes with the results contained in this work is that the Bloch Hamiltonian formalism that we have relied on necessarily gives rise to a two-band description. As a result, the applicability of our results may be limited under addition of more bands, especially when those bands are non-trivial---for example, four-band models incorporating both spin and orbital degrees of freedom may not be readily captured using cohomology in two or three dimensions. Defining more precisely what the scope of such descriptions is would be a great step forward in generality.

One of the biggest questions that merit more exploration with regards to the classification on $\KS$ is that relating to its surface states. In particular, while we have classified the surface states resulting from truncation in the periodic $S^1$ direction, the projections in the other two coordinate directions are more difficult to make sense of. In particular, it is worth exploring whether the $yz$ surface can be described using the unusual twist in both homology and cohomology that we have described, and how the groups should be calculated explicitly if this is the case.

On the topic of two-dimensional descriptions, the two-dimensional surface Mayer--Vietoris sequence in Equation~\eqref{eq:K2-MV-explicit} may find some applications outside of its original context of surface modes. Perhaps the most straightforward of these are the proper two-dimensional Weyl semimetals, which are usually protected by chiral symmetry \cite{Abdulla_chiral-WSM}. Class AIII 2D semimetals in particular have very similar topological features to those on the surface of 3D class A semimetals. Despite their apparent topological similarity, the two descriptions are substantially different: the surface of a three-dimensional Weyl semimetal hosts gapless Fermi arcs, while their topological equivalent in 2D class AIII features bulk Dirac strings which project onto Fermi arcs on the one-dimensional surfaces. Finding a canonical topological mapping between these two types of systems may be an important step in the full classification of two-dimensional Weyl semimetals. Incidentally, the two-dimensional chiral Weyl semimetals in classes BDI and CII both feature orientation-preserving symmetries which nevertheless invert Weyl point chiralities, and just as on the $yz$ surface of $\KS$, a description in terms of more generalised twisted coefficients may prove useful.

A somewhat more indirect, but at least as important application of the two-dimensional topology may come in the classification of certain non-Hermitian systems. This rapidly-growing field of physics uses non-Hermitian Hamiltonians in order to model systems with, e.g., physical gains and losses. Non-Hermitian systems feature special band degeneracies called \emph{exceptional points}, where not only the (complex) energy eigenvalues, but also their associated eigenstates coincide. These exceptional points have important features in common with Weyl points. Two works have recently gone into preprint which link this non-Hermitian physics quite directly to the work performed in this thesis. First, Lukas König, Kang Yang et al. have studied how exceptional points behave when the Brillouin zone is a two-dimensional non-orientable manifold \cite{KönigYang_nonorientable-EPs}. Similar features were observed to those of the Weyl points on $\KS$, including the reversal of chirality across certain boundaries of the Brillouin zone. Second, a work by Marcus Stålhammar and Lukas Rødland details how non-Hermitian systems with exceptional points can be canonically mapped to equivalent Hermitian Weyl semimetals, upon which they can be classified topologically using the same machinery we have employed here \cite{Staalhammar_EPn-cohomology}. Based on this description, it should be relatively straightforward to study the non-orientable features in Ref.~\cite{KönigYang_nonorientable-EPs} in terms of the analysis that we have already performed on Weyl semimetals.

Another aspect of our results that can be expanded upon is that of the three additional fundamental non-orientable Brillouin zones we have identified, i.e.\ those arising under the action of the space groups $Cc$, $Pca2_1$ and $Pna2_1$. We have provided a full classification of invariants for each of these spaces in terms of a Mayer--Vietoris sequence, but the physical interpretation of these invariants and the topology of the underlying Brillouin zone can certainly be elucidated further. As we have demonstrated in this work, twisted homology tends to provide an intuitive platform for the study of these invariants, and so we recommend that the relevant homology description be given attention especially. The space group $Pna2_1$ seems to give rise to particularly promising topology, being classified by a single $\Z_4$ insulating invariant. The insulating invariants of these spaces may also be probed experimentally in the context of acoustic crystals; as we have reviewed, the gauge fluxes necessary to realise momentum-space non-symmorphic symmetries can be implemented with ease there. The precise projective representations that these fluxes need to enact to give rise to each of these symmetries can be found listed in the supplement of Ref.~\cite{Zhang_nonsymmorphic}.

On the topic of acoustic crystals, Weyl semimetals have been successfully reproduced experimentally in these materials \cite{Xiao_acoustic-Weyl,Hao_acoustic-Weyl,Wang_acoustic-Weyl,Xiao_acoustic-semimetal}. Combining these semimetal realisations with gauge fluxes may provide a platform to study semimetal physics directly under non-symmorphic symmetries. Veering into speculation for a moment, being able to study the transport properties of such materials may help confirm that no net chiral anomaly appears at sufficiently large scales.

Finally, it may be worth exploring whether there are other symmetries besides inversion for which a similarly heuristic line of reasoning to that given in Section~\ref{sec:inversion-ansatz} can be given. Alternatively, it may be possible to further develop a formalism for treating high-symmetry points in symmetric semimetals. Such a formalism would allow for rapid calculation and interpretation of invariants for any given orientation-reversing symmetry.