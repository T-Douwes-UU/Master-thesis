\markedchapter{Topological states}{Topological states of matter \& symmetries}

\blue{Sources:\cite{Akhmerov_online-course}\cite{Asboth_topo-course}\cite{Bernevig_topological-insulators}}

\red{Finish intro when chapter is more complete} %TODO

\section{Basic definitions}
{\color{blue}
\begin{itemize}
	\item Conductive properties of materials are understood in terms of band structure → Fermi energy. Conductance means Fermi level lies inside one of the bands. [picture]
	
	\item $N$-band system has hilbert space $\Hc\cong\C^N$, Hamiltonian represented by $N\times N$ matrix. Static system: $H\psi = E\psi$, eigenvalues are energy bands.
	
	\item Mostly interested in 2-band systems since only valence/conduction bands are relevant. Then $H$ is a $2\times 2$ Hermitian (for now) matrix. These are given by $H = h_0\mathbb{I} + \h\cdot\bm{\upsigma}$ in general; $h_0$ corresponds to the Fermi level and can be normalized to 0. [understand this better] → Bloch Hamiltonian [higher dimensional systems: Clifford algebra]
	
	\item For a Bloch Hamiltonian, eigenvalues are $\pm\abs{\h}$, so conductance occurs when $\h = 0$.
	
	\item Insulating Hamiltonians are adiabatically connected if they can be continuously deformed into each other without band crossings. Insulators are considered topological if they are not adiabatically connected to a reference trivial phase; then these inhabit different regions of the phase diagram → existence of edge states (not always\cite{Bernevig_topological-insulators}, footnote)
\end{itemize}
}

\subsection{Bloch theory}
{\color{blue}
\begin{itemize}
	\item We work with crystalline materials which are composed of periodically repeating unit cells.
	
	\item In the bulk, we assume the Hamiltonian is periodic in the unit cell. This enables use of Bloch's theorem\cite{Bloch_theorem} $\psi(\vb{r}) = \e^{i\k\cdot\vb{r}}u_{\k}(\vb{r})$.
	
	\item Different values of crystal momentum may yield identical eigenstates, the set of equivalence classes is the Brillouin zone
	
	\item Brillouin zone usually has $\T^n$ topology, but internal symmetries etc. may alter this\cite{Foncesca-Vaidya_nonorientable} [other sources]
\end{itemize}
}


\section{One-dimensional models}
{\color{blue}
\begin{itemize}
	\item SSH is usually introduced "physics first", but we would like to work backwards in a sense, to see how bulk topology gives rise to physical properties of a system.
\end{itemize}
}

\subsection{The Su--Schrieffer--Heeger model}
{\color{blue}
\begin{itemize}
	\item Start with an infinite 1D chain of unit cells; then the Brillouin zone has $S^1\cong [-\pi,\pi]/\set{-\pi\sim\pi}$ topology
	
	\item Parametrizing the Hamiltonian by $\h(k)$ gives a map $\h:S^1\to\R^3$, and restricting to insulators gives $\h:S^1\to\R^3\setminus\set{0}$.
	
	\item However, $\pi_1(\R^3\setminus\set{0}) = 0$ and so all insulating Hamiltonians are adiabatically connected
	
	\item We can fix this by imposing $h_z = 0$, so that effectively $\h:S^1\to\R^2\setminus\set{0}$, with fundamental group $\Z$ indexed by winding number around the origin. We will see that this amounts to imposing a physical symmetry.
	
	\item Start with the simplest topologically trivial state $\h_{\rm triv}(k) = (1, 0, 0)\tran$ and the simplest topologically interesting state $\h_{\rm top}(k) = \big(\cos(k), \sin(k), 0\big)\tran$. We choose these combinations of $x$, $y$, $z$ conveniently to arrive at the SSH but in principle any similar model would be equivalent by change of basis.
	
	\item Now make a linear combination $\h(k) = v\h_{\rm triv}(k) + w\h_{\rm top}(k)$. This is trivial when $v>w$, conducting when $v=w$ and topological when $v<w$. [picture]
	
	\item The momentum-space Hamiltonian is now
	\[
		H(k) = \begin{pmatrix}
			0 & v + w\e^{-ik} \\
			v + w\e^{ik} & 0
		\end{pmatrix}.
	\]
	For FT we can suggestively rewrite as
	\[
		H(k) = \e^{-ik(n-n)}\begin{pmatrix}
			0 & v \\
			v & 0
		\end{pmatrix} + \e^{-ik\big((n+1)-n\big)}\begin{pmatrix}
		0 & w \\
		0 & 0
		\end{pmatrix} + \e^{-ik\big(n-(n+1)\big)}\begin{pmatrix}
		0 & 0 \\
		w & 0
		\end{pmatrix}
	\]
	where $n$ is the unit cell index.
	
	\item It follows [how exactly?] that we can write the Hamiltonian in a unit cell basis as
	\[
		\hat{H} = \sum_{n=-\infty}^{\infty}\left[\ket{n}\bra{n}\otimes\begin{pmatrix}
			0 & v \\
			v & 0
		\end{pmatrix} + \left(\ket{n+1}\bra{n}\otimes\begin{pmatrix}
		0 & w \\
		0 & 0
		\end{pmatrix} + {\rm h.c.}\right)\right]
	\]
	
	\item We can introduce a boundary by setting finite $N$ with open BC:
	\[
		\hat{H} = \sum_{n=0}^{N}\ket{n}\bra{n}\otimes\begin{pmatrix}
			0 & v \\
			v & 0
		\end{pmatrix} + \sum_{n=0}^{N-1}\left(\ket{n+1}\bra{n}\otimes\begin{pmatrix}
			0 & w \\
			0 & 0
		\end{pmatrix} + {\rm h.c.}\right)
	\]
	
	\item This case can be represented as
	\[
		\hat{H} = \begin{pNiceMatrix}
			\Block[borders={bottom,right,tikz=dashed}]{2-2}{}
			             0 & v & \Block[borders={bottom,right,tikz=dashed}]{2-2}{}
			                     0 & 0 & \Block{4-4}{0} &        &   & \\
			             v & 0 & w & 0 &                &        &   & \\
			\Block[borders={bottom,right,tikz=dashed}]{2-2}{}
			             0 & w & 0 & v &                &        &   & \\
			             0 & 0 & v & 0 &                &        &   & \\
			\Block{4-4}{0} &   &   &   &                & \Ddots & 0 & 0 \\
			               &   &   &   & \Ddots         &        & w & 0 \\
			               &   &   &   &              0 & w      & 0 & v \\
			               &   &   &   &              0 & 0      & v & 0
		\end{pNiceMatrix}
	\]
	
	\item This looks like a chain of length 2N with alternating hopping amplitudes, meaning we can realize the system in terms of unit cells with two atoms, intra-cell hopping $v$ and inter-cell hopping $w$. This is realised in e.g. polyacetylene: \includegraphics[width=.8\textwidth]{Images/polyacetylene}
	
	\item The trivial $v>w$ case has strong hopping within the cells, including to and from the edge modes. The conducting $v=w$ phase is just a chain of equal hoppings, and the topological $v<w$ has weaker hopping inside the cells so that the edge modes become isolated.
	
	\item We can now physically interpret the meaning of setting $h_z = 0$: it ensures that hopping only occurs between the two sublattices $A$ and $B$, and not within them (i.e.\ there are only off-diagonal elements in the internal degrees of freedom). If we define the sublattice projection operators
	\[
		\hat{P}_A = \mathbb{I} \otimes \begin{pmatrix}
			1 & 0 \\ 0 & 0
		\end{pmatrix},\quad \hat{P}_B = \mathbb{I} \otimes \begin{pmatrix}
		0 & 0 \\ 0 & 1
		\end{pmatrix}
	\]
	then the Hamiltonian obeys
	\[
		\hat{P}_A\hat{H}\hat{P}_A = \hat{P}_B\hat{H}\hat{P}_B = 0
	\]
	and so since $\hat{P}_A + \hat{P}_B$ is the identity we have
	\begin{align*}
		\hat{H} &= (\hat{P}_A + \hat{P}_B)\hat{H}(\hat{P}_A + \hat{P}_B) \\
			&= \hat{P}_A\hat{H}\hat{P}_B + \hat{P}_B\hat{H}\hat{P}_A \\
			&= (\hat{P}_A - \hat{P}_B)\hat{H}(\hat{P}_B - \hat{P}_A) \\
			&\equiv -\hat{\Gamma}\hat{H}\hat{\Gamma}
	\end{align*}
	with $\hat{\Gamma}\equiv\hat{P}_A - \hat{P}_B$ having the property that $\hat{\Gamma} = \hat{\Gamma}^{-1} = \hat{\Gamma}\dagger$; this is called sublattice symmetry and it also applies to the momentum space Hamiltonian $H(k)$.
	
	\item An immediate consequence of our setup is that the trivial and topological phase become adiabatically connected if we allow for sublattice symmetry breaking ($h_z \neq 0$).
\end{itemize}
}
%TODO write about boundary states
\red{Expand more on boundary states}

\subsection{The Kitaev chain}


\section{Two-dimensional models}

\subsection{The Kane--Mele model}

\subsection{Quantum Hall effect}