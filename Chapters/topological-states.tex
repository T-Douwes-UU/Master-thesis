\chapter[Topological states of matter]{Topological states of matter \& symmetries}

\blue{Sources:\cite{Akhmerov_online-course}\cite{Asboth_topo-course}\cite{Bernevig_topological-insulators}}

\red{Finish intro when chapter is more complete} %TODO

\section{Basic definitions}
{\color{blue}
\begin{itemize}
	\item Conductive properties of materials are understood in terms of band structure → Fermi energy. Conductance means Fermi level lies inside one of the bands. [picture]
	
	\item $N$-band system has hilbert space $\Hc\cong\C^N$, Hamiltonian represented by $N\times N$ matrix. Static system: $H\psi = E\psi$, eigenvalues are energy bands.
	
	\item Mostly interested in 2-band systems since only valence/conduction bands are relevant. Then $H$ is a $2\times 2$ Hermitian (for now) matrix. These are given by $H = h_0I + \vb{h}\cdot\bm{\upsigma}$ in general; $h_0$ corresponds to the Fermi level and can be normalized to 0. → Bloch Hamiltonian (higher dimensional systems?)
	
	\item For a Bloch Hamiltonian, eigenvalues are $\pm\abs{\vb{h}}$, so conductance occurs when $\vb{h} = 0$.
	
	\item Insulating Hamiltonians are adiabatically connected if they can be continuously deformed into each other without band crossings. Insulators are considered topological if they are not adiabatically connected to a reference trivial phase; then these inhabit different regions of the phase diagram → existence of edge states (not always\cite{Bernevig_topological-insulators}, footnote)
\end{itemize}
}

\subsection{Bloch theory}
{\color{blue}
\begin{itemize}
	\item We work with crystalline materials which are composed of periodically repeating unit cells.
	
	\item In the bulk, we assume the Hamiltonian is periodic in the unit cell. This enables use of Bloch's theorem\cite{Bloch_theorem}
\end{itemize}
}


\section{One-dimensional models}

\subsection{The Su--Schrieffer--Heeger model}

\subsection{The Kitaev chain}


\section{Two-dimensional models}

\subsection{The Kane--Mele model}

\subsection{Quantum Hall effect}